\chapter*{Introduction}
\label{cap:introduction}
\addcontentsline{toc}{chapter}{Introduction}

\chapterquote{La revolución industrial y sus consecuencias han sido un desastre para la raza humana}{Theodore Kaczynski}

In recent years, there has been a great interest and evolution of virtual reality technologies led by companies such as Apple with the launch of Apple Glasses or Meta with the launch of Meta Quest 3 or its interest in the \gls{metaverse} with its application ``Meta Horizon Worlds''.


It is for this reason that it is interesting to investigate non-verbal communication in virtual environments, not only for applications with multiple users but also to improve interactions with \glspl{npc} in these types of environments.


The proposal in this Bachelor's Thesis is a first approach to how we can use Artificial Intelligence and motion capture technology to achieve this improvement in interactions in virtual worlds through non-verbal communication, being our case the exploration of the ability to classify different gestures.
The gestures we have focused on are six gestures that we have considered useful for an NPC to recognize in a video game. These gestures are:
\begin{enumerate}
	%\renewcommand{\theenumi}{\alph{enumi}}
	\item Dance
	\item Greet
	\item Point
	\item Sit down
	\item Fight
	\item Run
\end{enumerate}

To carry out this work, we have used the Meta (formerly Oculus) Quest 2 and 3 virtual reality glasses and the Perception Neuron 3 motion capture suit, from the company Noitom, as hardware and Python, C\# and Unity as the software requirements.
\section{Motivation}
Research on gesture recognition using \gls{ai} for possible implementations in the study and improvement of non-verbal communication in virtual environments.

\section{Objectives}
Implementation of an \gls{ai} model that, with low latency, allows identifying the gesture being performed with a motion capture suit.

\section{Work Plan}
The Work Plan consists of several steps:

\begin{enumerate}
	%\renewcommand{\theenumi}{\alph{enumi}}
	\item Search for a dataset: generate a dataset large enough with several examples of gestures to adequately train different models.
	\item Implementation of \gls{ai} models: implementation of several \gls{ai} models to make a comparison between them and decide which one is the most suitable considering its prediction speed and accuracy.
	\item Development of a final application: development of an application for the Meta Quest as a demo that connects to the chosen model and allows real-time usage.
\end{enumerate}









