\chapter{Conclusiones y Trabajo Futuro}
\label{cap:conclusiones}

\section{Conclusiones}

Este estudio presenta la comparativa de tres redes neuronales básicas y un \gls{randomforest} para la clasificación de poses realizadas por un humano llevando un traje de captura de movimiento. Teniendo en cuenta las limitaciones del traje (las personas con algún tipo de discapacidad del aparato motor, por ejemplo la falta de alguna extremidad o miembro, no pueden usar correctamente el traje), se ha conseguido un modelo relativamente ligero que detecta las poses de todo el dataset con un gran nivel de acierto.

Este estudio también ha tenido sus limitaciones, como la falta de datos de entrenamiento, la falta de precedentes en el uso de trajes de captura de movimiento para entrenar modelos de \gls{ia} y el uso de hardware de consumidor en vez de uno más profesional. Todos estos problemas se han ido subsanando con la ayuda de los voluntarios de los experimentos, el conocimiento adquirido durante la carrera y la ayuda de la asociación LAG al ceder equipos para el entrenamiento.

Este estudio puede servir para la investigación de otros modelos que se centren más en las diferencias emocionales entre mismos tipos de gesto, diferencias culturales entre distintos tipos de gestos o otros temas relacionados con movimientos humanos.



\section{Trabajo Futuro}

El trabajo a futuro de este estudio se puede centrar en varios esfuerzos:
\begin{itemize}
    \item Mejorar el dataset, incluyendo gente con diversidad funcional, aumentando el número de muestras por gesto y añadiendo nuevas categorías.
    \item Investigar otros modelos de \gls{ia} que puedan mejorar la precisión de la clasificación y hacer clasificación no solo de gestos, sino también de emociones.
    \item Estandarizar el servidor para que pueda ser usado con otras tecnologías de \gls{ia} y no solo con TensorFlow.
\end{itemize}



