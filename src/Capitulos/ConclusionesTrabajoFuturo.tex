\chapter{Conclusiones y Trabajo Futuro}
\label{cap:conclusiones}

\section{Conclusiones}

% Este estudio presenta la comparativa de tres redes neuronales básicas y un \gls{randomforest} para la clasificación de poses realizadas por un humano llevando un traje de captura de movimiento. Teniendo en cuenta las limitaciones del traje (las personas con algún tipo de discapacidad del aparato motor, por ejemplo la falta de alguna extremidad o miembro, no pueden usar correctamente el traje), se ha conseguido un modelo relativamente ligero que detecta las poses de todo el dataset con un gran nivel de acierto.

% Este estudio también ha tenido sus limitaciones, como la falta de datos de entrenamiento, la falta de precedentes en el uso de trajes de captura de movimiento para entrenar modelos de \gls{ia} y el uso de hardware de consumidor en vez de uno más profesional. Todos estos problemas se han ido subsanando con la ayuda de los voluntarios de los experimentos, el conocimiento adquirido durante la carrera y la ayuda de la asociación LAG al ceder equipos para el entrenamiento.

% Este estudio puede servir para la investigación de otros modelos que se centren más en las diferencias emocionales entre mismos tipos de gesto, diferencias culturales entre distintos tipos de gestos o otros temas relacionados con movimientos humanos.

\subsection{Conclusiones de la búsqueda de datasets}
Tras la búsqueda de datasets de animaciones se ha llegado a la conclusión de que no existen datasets públicos suficientes para el objetivo de este estudio. Esta escasez se debe no solo a la falta en número de datasets, sino también a la falta de gestos referentes a comunicación no verbal, habiendo algunos sobre deportes o temáticas muy concretas y pocos de carácter general.  Otra limitación de estos datasets ha sido la facilidad de adecuación o conversión a un formato compatible con el traje de captura de movimiento usado en el estudio.

Estas conclusiones nos llevaron a la necesidad de crear un dataset propio recabando animaciones de bancos de animaciones y mediante pruebas con usuarios.

\subsection{Conclusiones de la extracción de animaciones de bancos de animaciones}
Para la conversión de animaciones de Mixamo se ha creado una herramienta para hacerlo de forma automática y se ha modificado otra herramienta para hacer una descarga en masa de estas animaciones.

La herramienta de conversión de gestos cumple su objetivo completamente, ya que tan solo indicando el nombre de la carpeta que contiene los gestos descargados coge los gestos, se procesan en Unity y se suben a Kaggle de forma automática.
Además la herramienta es independiente al número de gestos que se quieren convertir y a su vez del número de animaciones que hay de esos gestos, consiguiendo una gran escabilidad de esta forma haciendo posible la introducción de nuevos gestos sin tener que cambiar nada de la herramienta.

Otra conclusión que se ha sacado a partir de la extracción de animaciones en Mixamo es que los datos obtenidos estaban desbalanceados en número, haciendo necesaria la recolección de datos con usuarios.

\subsection{Conclusiones de la recolección de animaciones con usuarios}
Para la recolección de datos con usuarios se anunció de forma insistente en las redes sociales de los autores del estudio así como por las distintas facultades de la Universidad Complutense de Madrid.
Los anuncios cumplieron su objetivo, consiguiendo 75 respuestas de los cuales se presentaron 65 participantes, consiguiendo un 86,67\% de asistencia.

Como se puede ver en las figuras del apéndice \ref{appendix:formularioDemografia} el grupo de participantes, con excepción del género no binario, es heterogéneo en cuanto al género con una pequeña diferencia de porcentaje entre hombres y mujeres.
Donde se puede ver mayor diferencia es entre la mano dominante de los usuarios, siendo los zurdos tan solo un 6.15\% de la población total del grupo.

Gracias a la prueba se obtuvieron 975 animaciones (195 animaciones de saludar, señalar, sentarse, pelear y correr), lo que se pensó que solucionaría el desbalanceo con respecto al baile, pero se después de la estandarización de los datos se comprobó que la animación de bailar seguía estando bastante diferenciada del resto.
Esto se puede ver en las figuras \ref{fig:datos-bruto} y \ref{fig:datos-estandar}.

\subsection{Conclusiones de la comparativa entre los modelos de \gls{ia}}

En cuanto a los modelos de \gls{ia}, se ha visto que los modelos de redes no han obtenido los resultados esperados, siendo esto seguramente por una mezcla de falta de datos y simplificación de las redes neuronales para asegurar un tiempo de entrenamiento menor y un computo más rápido a la hora de realizar inferencias.

Sin embargo, el \gls{randomforest} ha sido el que ha obtenido mejores resultados, siendo el modelo más rápido (tanto en inferencia como en entrenamiento), el modelo más ligero y el modelo más adaptable. Este modelo se podría usar en sistemas con recursos limitados como pueden ser gafas de \gls{vr} independientes como las Meta Quest o en ordenadores más potentes, e incluso en ecosistemas de \gls{ia} por su posible traducción a TensorFlow ya implementada.

En el análisis de los resultados del \gls{randomforest} se ha visto que el modelo le da una gran importancia a la coordenada y de la pierna izquierda, a la coordenada y del pie izquierdo en el segundo intervalo de tiempo. Esto puede indicar que los gestos dependen mucho de la posición de las piernas para discriminar si se está corriendo, bailando, saludando o señalando.

También se puede ver una ligera confusión entre los gestos de pelea y saludar. Esto puede ser debido a los movimientos rápidos de las manos en ambos tipos de gesto, que al no tener dedos por limitación del traje, puede dar lugar a confusión entre un puñetazo y un saludo.

\subsection{Conclusiones de la aplicación final}
La aplicación final es una demo que se conecta a un servidor para mostrar los resultados.

La parte negativa de esta aplicación es la forma de conectarse con el sevidor en el que está el modelo, ya que hace peticiones web al servidor de Tensor Serving directamente, lo que lo hace dependiente de éste.

Finalmente la aplicación final cumple su propósito de enseñar al usuario la animación predicha con una latencia mínima.

\subsection{Conclusiones generales}
El trabajo ha cumplido con los objetivos propuestos, haciendo posible la detección de gestos en tiempo real con un pequeño márgen de error y la publicación de un dataset de animaciones captadas por usuarios reales.

\section{Trabajo Futuro}

El trabajo a futuro de este estudio se puede centrar en varios esfuerzos:
\begin{itemize}
    \item Mejorar el dataset, incluyendo gente con diversidad funcional, aumentando el número de muestras por gesto, teniendo en cuenta los dedos de las manos y añadiendo nuevas categorías.
    \item Investigar otros modelos de \gls{ia} que puedan mejorar la precisión de la clasificación y hacer clasificación no solo de gestos, sino también de emociones.
    \item Estandarizar el servidor para que pueda ser usado con otras tecnologías de \gls{ia} y no solo con TensorFlow.
    \item Implementar otro modelo que no solo tenga en cuenta el gesto predicho actual sino también todo el contexto para poder avanzar en el uso de la comunicación no verbal con \glspl{npc}
\end{itemize}



