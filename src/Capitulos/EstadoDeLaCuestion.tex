\chapter{Estado de la Cuestión}
\label{cap:estadoDeLaCuestion}
En este capítulo se presenta el estado del arte de los distintos elementos que componen el trabajo. Estos campos son la comunicación no verbal en entornos virutales, la captura de movimiento, distintos modelos de \gls{ia}, motores de videojuegos (en concreto Unity) y el estado actual de los entornos virtuales.


%Dejo esto por aquí a ver qué te parecen estos apartados para hablar de ellos
\section{Comunicación no verbal en entornos virtuales}
En los últimos años se han llevado a cabo distintos estudios sobre la comunicación no verbal en entornos virtuales. El estudio \cite{XGKD23} se centra en el concepto de presencia en entornos virtuales, enfatizando los roles de inmersión, contenido emocional y fidelidad de avatares en la mejora de la experiencia del usuario. El estudio resalta como el realismo de los avatares y sus comportamientos (movimiento y lenguaje corporal) ayudan a conseguir un mayor sentimiento de presencia social entre los usuarios. También se discute como estos elementos influyen en las dinámicas interpersonales  y la percepción del usuario en entornos virtuales.
\section{Captura de movimiento}
\section{Modelos de \gls{ia}}
\section{Motores de videojuegos}
\section{Entornos virtuales}
