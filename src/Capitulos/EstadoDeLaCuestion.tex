\chapter{Estado de la Cuestión}
\label{cap:estadoDeLaCuestion}
En este capítulo se presenta el estado del arte de los distintos elementos que componen el trabajo. Estos campos son la comunicación no verbal en entornos virtuales, la captura de movimiento, distintos modelos de \gls{ia}, motores de videojuegos (en concreto Unity) y el estado actual de los entornos virtuales.


%Dejo esto por aquí a ver qué te parecen estos apartados para hablar de ellos
\section{Comunicación no verbal en entornos virtuales}
En los últimos años se han llevado a cabo distintos estudios sobre la comunicación no verbal en entornos virtuales. El estudio \cite{XGKD23} se centra en el concepto de presencia en entornos virtuales, enfatizando los roles de inmersión, contenido emocional y fidelidad de avatares en la mejora de la experiencia del usuario. El estudio resalta como el realismo de los avatares y sus comportamientos (movimiento y lenguaje corporal) ayudan a conseguir un mayor sentimiento de presencia social entre los usuarios. También se discute como estos elementos influyen en las dinámicas interpersonales  y la percepción del usuario en entornos virtuales.

Estos roles discutidos en el estudio pueden extrapolarse a la experiencia de jugar a videojuegos. En el contexto de los videojuegos tradicionales, las opciones de comunicación no verbal son las dadas por los desarrolladores en forma de gestos, acciones dentro del juego (agacharse, girar su avatar o mover la cabeza al mover la cámara por ejemplo). Dentro de cada videojuego la comunidad va a intentar usar las opciones que se les proporcione para comunicarse entre ellos mismos de la mejor manera posible.

Un ejemplo claro de evolución de la comunicación no verbal en videojuegos y su importancia para los jugadores es el caso de \textit{VR chat}. \textit{VR chat} es un videojuego \gls{mmorpg} de \gls{vr} donde los jugadores pueden unirse a distintos mundos e interactuar con otros usuarios. Este juego no solo permite el uso de gafas de \gls{vr}, sino que también permite (de manera experimental) usar distintas formas de captura de movimiento de cuerpo completo \footnote{Enlace a la documentación de \textit{VR Chat} para su \gls{sdk} de captura de movimiento de cuerpo completo \cite{VRCHATSDK}}.

Los usuarios de esta aplicación han buscado distintas formas de conseguir una mayor expresividad en sus avatares para así poder entablar mejor conversaciones entre ellos o para poder expresarse como les gustaría. Algunos de estos métodos se describen en el siguiente apartado junto a métodos de captura de movimiento no orientados a videojuegos.

\section{Captura de movimiento}

La captura de movimiento es una técnica que permite digitalizar el movimiento de una persona. Esta técnica se usa sobre todo en el ámbito de la animación, el cine y los videojuegos sobre todo. HABLAR DE  TRAJES DE CAPTURA DE MOVIMIENTO CON LUCES, DE TRAJES COMO EL PN3 Y DE LOS FURROS CON ONDAS CEREBRALES.

\section{Motores de videojuegos}


\section{Modelos de \gls{ia}}

La inteligencia artifical ha sido un tópico de desarrollo e investigación en los útlimos años. Aunque el desarrollo en los últimos años se ha centrado ha centrado sobre todo en el ámbito de las \glspl{ia} generativas, este estudio se va a centrar sobre todo en el uso de redes neuronales e \glspl{ia} de clasificación más tradicionales.

\subsection{Tecnologías de \gls{ia}}

En la actualidad, el mundo de la \gls{ia} está centrado en el desarrollo en python mediante el uso de librerías programadas en C++. Las librerías más comunes son Tensorflow\footnote{Enlace a la página principal de Tensorflow \url{https://www.tensorflow.org/}}, Pytorch\footnote{Enlace a la página principal de Pytorch \url{https://pytorch.org/}} y scikit-learn\footnote{Enlace a la página principal de Scikit-learn \url{https://scikit-learn.org/stable/}}.

Tensorflow tiene un enfoque más orientado a la producción y despliegue de modelos de \gls{ia} a un nivel empresarial, pytorch es más usado en el ámbito académico y de investigación por su capacidad de rápido prototipado. Scikit-learn se usa más para enseñar los principios de la \gls{ia} y desarrollar modelos más ligeros y más tradicionales.

Estas librerías se usan en conjunto con Numpy\footnote{Enlace a la página principal de Numpy: \url{https://numpy.org/}} y Pandas\footnote{Enlace a la página principal de Pandas: \url{https://pandas.pydata.org}} para conseguir unos cálculos más efectivos sobre la gran cantidad de datos que se necesitan en el ciclo de vida de un modelo de \gls{ia}.

\subsection{Tensorflow vs Pytorch}

Las dos librerías poseen ventajas y desventajas que las hacen más o menos igual de viables para este proyecto. Mientras que Pytorch permite una rápida experimentación y prototipación, Tensorflow da facilidades a la hora de desplegar los modelos en formato \gls{api}.

Tensroflow ofrece facilidades como Tensorflow Serving\footnote{Enlace a la página de Tensorflow Serving \url{https://www.tensorflow.org/tfx/guide/serving}} y TensorBoard\footnote{Enlace a la página de Tensorboard \url{https://www.tensorflow.org/tensorboard}} para el despliegue y visualización de los modelos. Pytorch ofrece una librería llamada TorchServe\footnote{Enlace a la página de TorchServe \url{https://pytorch.org/serve/}} que permite el despliegue de modelos, pero no tiene una herramienta como Tensorboard para la visualización de los modelos.

\subsection{YDF vs Scikit-learn}

Scikit-learn e YDF(\cite{GBBSP23}) comparten un punto en común: los árboles de decisión. Los árboles de decisión son modelos de aprendizaje automático no supervisado usados para tareas de clasificación y regresión. Estos modelos se basan en la idea de dividir distintos grupos a partir de distintos criterios dados. Estas divisiones se hacen en forma de árbol binario, dando así razón a su nombre.

\section{Entornos virtuales}
Como se puede ver en el artículo \cite{NOVE} se empezó a hablar en la década de los 90 de entornos virtuales y el \gls{metaverso} como nuevas formas de comunicarse con las personas.
Estos entornos virtuales al principio eran muy sencillos, con pocas actividades que realizar y el chat escrito como única forma de comunicarse. Algún ejemplo de estos entornos virtuales eran el Second Life o el Habbo Hotel.

Pero con la aparición de la \gls{vr} se hicieron populares entornos virtuales focalizados en esta tecnología, ya que ofrecía un grado de interacción superior a los anteriores mencionados.
Junto a esto se empieza a hablar de ``inmersión'' y ``presencia'': dos términos que, a menudo se usan como sinónimos, pero son ligeramente distintos aunque están muy relacionados.
Como lo define el artículo \cite{MRPI}, la presencia es la calidad experimental en entornos virtuales mientras que la inmersión está asociado con los aspectos técnicos de un sistema virtual que ayudan al usuario a potenciar la sensación de presencia.
Estudios recientes como \cite{FPPS} demuestran como la presencia, tanto la nuestra propia como social, son potenciadores del apoyo social percibido, lo que se asocia con el bienestar de los usuarios; mientras que el estudio \cite{GVR} muestra que, aunque no haya mejoras performáticas entre jugar en entornos inmersivos (\gls{vr}) y no inmersivos (un ordenador), los jugadores encontraron más emocionantes y con mayor presencia los entornos interactivos.

Con todo esto es natural que las empresas inviertan en tecnologías que aumenten la sensación de inmersión para lograr una mayor presencia.
Empresas como Apple o Meta están apostando en la tecnología \gls{vr}, sacando recientemente las gafas Apple Vision Pro (2024) y Meta Quest 3 (2023); mejorando aspectos como la calidad de las cámaras, el sonido, el reconocimiento de gestos y su comodidad que pueden amplificar la sensación de inmersión cuando se utilizan.
Además, Meta está centrada en la idea de \gls{metaverso} como un espacio digital en el que se puede socializar.

Con la idea de mejorar la inmersión en entornos virtuales se ideó este proyecto en el que, mediante la captura de movimiento, se puede usar la comunicación no verbal para expresar ciertas acciones y el entorno sea capaz de reconocer lo que estamos haciendo. Todo esto pensado para que se pueda usar en entornos de \gls{vr} para llevar la inmersión al máximo

