\chapter{Estado de la Cuestión}
\label{cap:estadoDeLaCuestion}
En este capítulo se presenta el estado del arte de los distintos elementos que componen el trabajo. Estos campos son la comunicación no verbal en entornos virutales, la captura de movimiento, distintos modelos de \gls{ia}, motores de videojuegos (en concreto Unity) y el estado actual de los entornos virtuales.


%Dejo esto por aquí a ver qué te parecen estos apartados para hablar de ellos
\section{Comunicación no verbal en entornos virtuales}
En los últimos años se han llevado a cabo distintos estudios sobre la comunicación no verbal en entornos virtuales. El estudio \cite{XGKD23} se centra en el concepto de presencia en entornos virtuales, enfatizando los roles de inmersión, contenido emocional y fidelidad de avatares en la mejora de la experiencia del usuario. El estudio resalta como el realismo de los avatares y sus comportamientos (movimiento y lenguaje corporal) ayudan a conseguir un mayor sentimiento de presencia social entre los usuarios. También se discute como estos elementos influyen en las dinámicas interpersonales  y la percepción del usuario en entornos virtuales.

Estos roles discutidos en el estudio pueden extrapolarse a la experiencia de jugar a videojuegos. En el contexto de los videojuegos tradicionales, las opciones de comunicación no verbal son las dadas por los desarrolladores en forma de gestos, acciones dentro del juego (agacharse, girar su avatar o mover la cabeza al mover la cámara por ejemplo). Dentro de cada videojuego la comunidad va a intentar usar las opciones que se les proporcione para comunicarse entre ellos mismos de la mejor manera posible.

Un ejemplo claro de evolución de la comunicación no verbal en videojuegos y su importancia para los jugadores es el caso de \textit{VR chat}. \textit{VR chat} es un videojuego \gls{mmorpg} de \gls{vr} donde los jugadores pueden unirse a distintos mundos e interactuar con otros usuarios. Este juego no solo permite el uso de gafas de \gls{vr}, sino que también permite (de manera experimental) usar distintas formas de captura de movimiento de cuerpo completo \footnote{Enlace a la documentación de \textit{VR Chat} para su \gls{sdk} de captura de movimiento de cuerpo completo \cite{VRCHATSDK}}.

Los usuarios de esta aplicación han buscado distintas formas de conseguir una mayor expresividad en sus avatares para así poder entablar mejor conversaciones entre ellos o para poder expresarse como les gustaría. Algunos de estos métodos se describen en el siguiente apartado junto a métodos de captura de movimiento no orientados a videojuegos.

\section{Captura de movimiento}

La captura de movimiento es una técnica que permite digitalizar el movimiento de una persona. Esta técnica se usa sobre todo en el ámbito de la animación, el cine y los videojuegos sobre todo. HABLAR DE  TRAJES DE CAPTURA DE MOVIMIENTO CON LUCES, DE TRAJES COMO EL PN3 Y DE LOS FURROS CON ONDAS CEREBRALES.

\section{Motores de videojuegos}


\section{Modelos de \gls{ia}}


\section{Entornos virtuales}
