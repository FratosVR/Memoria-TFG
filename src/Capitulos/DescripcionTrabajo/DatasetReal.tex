\section{Dataset real}
\label{sec:datasetReal}
Debido a que las animaciones encontradas en el banco de animaciones no eran las suficientes y estaban descompensadas (TO DO: METER GRÁFICA DE LA CANTIDAD DEL DATASET Y QUE SE VEA LA DESCOMPENSACIÓN) se decidió crear una herramienta para poder recoger los datos de usuarios.

(TO DO: HABLAR EN PROFUNDIDAD DE LA HERRAMIENTA. UN UML ESTARÍA BIEN)

Esta recogida de datos consistió en ponerle el traje de captura de movimiento a los usuarios y pedirles que realizasen tres tomas de todos los gestos, a excepción del gesto de baile, ya que de ese gesto teníamos basantes más datos que el resto.

Para ello se creó un formulario en el que usuarios interesados en ello se apuntasen. En el formulario se explicaba el objetivo de la prueba y se numeraban los datos que iban a ser pedidos en la prueba. Los campos a rellenar eran: 

\begin{enumerate}
	%\renewcommand{\theenumi}{\alph{enumi}}
	\item Correo
	\item Aceptar recogida de datos
	\item Posible hueco en un horario, en huecos de una hora de lunes a viernes y de 9:00 a 20:00
\end{enumerate}

Una vez creado el formulario se creó un cartel (TO DO: PONER FOTO DEL CARTEL) el cual se colgó en redes y se colgó en distintas facultades (NO SÉ SI PROCEDE COMENTARLO PERO BUENO), teniendo como resultado que se apuntasen 75 personas en el formulario.

Lo siguiente era tener un sitio en el que hacer las pruebas. Debido a que las pruebas requerían movimiento era preciso un lugar en el que se pudiesen mover sin dificultades y que no estuviese a la vista para preservar la intimidad de los usuarios.

TO DO: HABLAR DE LA RAMIFICACIÓN Y PODA HECHA PARA CONSEGUIR LOS HUECOS

Finalmente desde el día 14/042025 a las 9:00 hasta el día 17/04/2025 a las 19:30 se pudo grabar en el despacho (NO ME ACUERDO DEL NÚMERO) de la Facultad de Informática de la Universidad Complutense de Madrid.
De los 75 usuarios apuntados finalmente se presentaron 69, consiguiendo así 975 animaciones en formato CSV, 195 de cada gesto. (TO DO: PONER FOTOS DE LAS PRUEBAS Y LOS USUARIOS)

TO DO: HABLAR DE LOS DATOS PEDIDOS DE LOS USUARIOS Y SUS RESULTADOS, SUS ESTADÍSTICAS Y TODO ESO

TO DO: HABLAR DE LAS ANIMACIONES YA PROCESADAS EN QUÉ SE QUEDA Y DAR PASO A LA PARTE DE LOS MODELOS