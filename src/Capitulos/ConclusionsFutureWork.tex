\chapter*{Conclusions and Future Work}
\label{cap:conclusions}
\addcontentsline{toc}{chapter}{Conclusions and Future Work}
\section{Conclusions}

This study presents a comparison of three basic neural networks and a \gls{randomforest} for the classification of poses performed by a human wearing a motion capture suit. Taking into account the limitations of the suit (people with some type of motor disability, such as the absence of a limb, cannot use the suit properly), a relatively lightweight model has been achieved that detects the poses in the entire dataset with a high level of accuracy.

This study also had its limitations, such as the lack of training data, the lack of precedents in the use of motion capture suits to train \gls{ai} models, and the use of consumer-grade hardware instead of more professional equipment. All these issues were mitigated with the help of the experiment volunteers, the knowledge acquired during the degree, and the support of the LAG association by providing equipment for training.

This study can serve as a basis for the research of other models that focus more on emotional differences between the same types of gestures, cultural differences among different types of gestures, or other topics related to human movement.

\section{Future Work}

The future work of this study can focus on several efforts:
\begin{itemize}
    \item Improve the dataset by including people with functional diversity, increasing the number of samples per gesture, and adding new categories.
    \item Investigate other \gls{ai} models that could improve classification accuracy and enable the classification not only of gestures but also of emotions.
    \item Standardize the server so it can be used with other \gls{ai} technologies and not just TensorFlow.
\end{itemize}


