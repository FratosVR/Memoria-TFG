\chapter*{Resumen}

\section*{\tituloPortadaVal}

Este estudio se centra en la detección de gestos específicos mediante el traje de captura de movimiento Perception Neuron 3 con el fin de crear interacciones con \glspl{npc} a través de la comunicación no verbal.
Para ello se han conseguido datos artificiales a través del banco de animaciones Mixamo y reales a través de pruebas de usuarios (N=65) con el traje previamente mencionado.
Con los datos recogidos y estandarizados se ha procedido a hacer una comparativa según su precisión entre cuatro modelos de \gls{ia}: Long Short-Term Memory, Convolutional Neural Network, Recurrent Neural Network y Random Forest, siendo éste último el mejor modelo para el caso.
Con el modelo seleccionado se ha desplegado un servidor gracias a Tensor Serving para poder usar el modelo en una aplicación desarrollada a modo de demo con soporte para gafas de \gls{vr} para mostrar los resultados.


\section*{Palabras clave}

\noindent Captura de movimiento, Unity, Inteligencia Artificial, Comunicación no verbal, Perception Neuron




