\newacronym{ia}{IA}{Inteligencia Artificial}
\newacronym{csv}{CSV}{Comma Separated Values}
\newacronym{ai}{AI}{Artificial Intelligence}
\newacronym{vr}{VR}{Realidad Virtual}
\newglossaryentry{metaverso}
{name=metaverso,
    description={El metaverso es un conjunto de espacios digitales para socializar, aprender, jugar y realizar otras actividades. (Definición de la empresa Meta)}}
\newglossaryentry{metaverse}
{name=metaverse,
    description={The metaverse is a set of  digital spaces to socialize, learn, play and do other activities. (Definition by the company Meta)}}
\newacronym{npc}{NPC}{Non Playable Character}
\newglossaryentry{rig}{
    name=rig,
    description={Un rig es el resultado del rigging, una técnica usada en animación digital que permite crear estructuras alrededor de una malla para poder animar de forma más sencilla}
}
\newglossaryentry{FBX}{
    name=FBX,
    description={El formato FBX es un tipo de archivo que contiene datos de objetos en 3D así como de animaciones}
}
\newglossaryentry{Animator}{
    name=Animator,
    description={Componente de Unity que funciona como una máquina de estados para la ejecución de animaciones}
}
\newacronym{mmorpg}{MMORPG}{Massively Multiplayer Online Role-Playing Game}
\newacronym{sdk}{SDK}{Software Development Kit}
\newglossaryentry{API REST}{
    name = API REST,
    description ={Arquitectura de diseño utilizada para crear servicios web y comunicar diferentes sistemas}
}

\newglossaryentry{fork}{
    name = fork,
    description ={Copia de un proyecto, generalmente software, que se hace con la intención de ser modificado y desarrollado, con posibilidad de independencia del proyecto original o con afán de mejorar el original}
}
\newacronym{fps}{FPS}{Frames Per Second}

\newacronym{api}{API}{Application Programming Interface}